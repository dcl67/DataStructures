\documentclass{article}
\usepackage{amsmath}
\usepackage{algorithm2e}
\begin{document}


\title{CS 260: Homework 2}
\author{Daniel Lopez}
\maketitle

\date{8 July 2017}

\section{1}
\subsection{g1(n) and g3(n)}
$g1(n)$ is $O(g3(n))$ for all even n $>=$ 0.\\
$g1(n)$ is $\Omega(g3(n))$ for all odd n.\\
$g3(n)$ is $O(g1(n))$ for odd n $>=$ 1.\\
$g1(n)$ is $\Omega(g3(n))$ for all even n.\\ 
\subsection{g1(n) and g2(n)}
\par$g1(n)$ is $O(g2(n))$ for all even and odd n for n $>$ 100.\\
$g1(n)$ is $\Omega(g2(n))$ for all n except for even n for n $>$ 100.\\
$g2(n)$ is $O(g1(n))$ for all n except for n $>$ 100\\
$g2(n)$ is $\Omega(g1(n))$ for all n except for n $<$ 100.
\subsection{g2(n) and g3(n)}
\par $g2(n)$ is $O(g3(n))$ for all n 0 $<=$ n $<=$ 100.\\
$g2(n)$ is $\Omega(g3(n))$ for n n $>$ 100.\\
$g2(n)$ is $O(g2(n))$ for n $>$ 100.\\
$g3(n)$ is $\Omega(g2(n))$ for all n $<$ 100.\\

\section{2}
\subsection{a}
Since 17 is a constant, it is not dependent on any variable. For one input, there is one output: 17. With no other dependencies, 17 is $O(1)$.
\subsection{b}
By definition of $T(n)$ is $O(f(n))$:
$T(n) = \frac{n(n-1)}{2}$

\begin{align}
	2*\frac{n(n-1)}{2} <=&2*\frac{1}{2}n^2
	n(n-1) <=&n^2
	\frac{n^2-n}{n} <=& \frac{n^2}{n}
	n-1 <=& n
\end{align}

Thus, $\frac{n(n-1)}{2}$ is $O(n^2)$

\subsection{c}
By definition of $T(n)$ is $O(n^3)$
$T(n)$ = max($n^3$, $10(n^2)$)
$n = 11$
\begin{align}
	max( 11^3, 10(11^2 ) <=& 11^3\\
	max( 1331, 1210 ) <=& 1331\\
	1331 <=& 1331\\
\end{align}
Thus, max($n^3$, $10(n^2)$) is $O(n^3)$ for values of $n >= 10$.


\subsection{d}
By summing, we can prove that $\sum\limits_{i=1}^{n} i^{k}$ is $O(n^{k+1}$ and $\Omega(n^{k+1})$.
	$1^k + 2^k + $ ... $+ n^{k+1} <= n^{k+1}$
	$1^k + 2^k + $ ... $+ n^{k+1} >= n^{k+1}$
Since both of these fit the definition of equal, $\sum\limits_{i=1}^{n} i^{k}$ is $O(n^{k+1}$ and $\Omega(n^{k+1})$, or $\Theta(n^{k+1})$.

\subsection{e}
If $a$ is a positive integer, $a_{k}n^k + a_{k-1}n^{k-1} + ... + a_{0}$.
Clearly, $P(x) <= a_{k}n^k + a_{k-1}n^{k-1} + ... + a_{0}$, so $P(x)$ is $O(n^k)$.
Also, since $a_{k} >= 0$, $P(x)$ is $\Omega(n^k)$

\section{3}
$\frac{1}{3}^n < 17 < log(log(n)) < log(n) < log^{2}(n) < \sqrt{n} < \sqrt{n}log^{2}(n) < \frac{n}{log(n)} < n < \frac{2}{3}^{n}$

\section{4}
The max function is called twice and is of size n/2. This means we can use the master theorem:
$T(j) = 2T(\frac{n}{2}) + 1$
Solving using the master theorem:
\begin{align}
	a = 2
	b = 2
	c = log_{2}(n) = 1
	1 <= n
\end{align}
The max function satisfies case 1, the definition of which states that $2T(\frac{n}{2}) + 1$ is $O(n)$ and $\Omega(n)$.

\section{5}
The delete function doesn't work from line 4, where it should read\\
\begin{algorithm}[H]
while p $<$ END(L)
\end{algorithm}
Additionally, the function could immediately check if the array is empty with \\
\begin{algorithm}[H]
if not L:\;
	return L\; 
else\;
	delete ..."\;
\end{algorithm}

\section{6}
FIRST: $n^2$\\
END: $n^2$\\
NEXT: $n^3$\\

\end{document}
