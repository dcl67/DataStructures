\documentclass{article}

\usepackage{amsmath}
\usepackage{algorithm2e}
\usepackage{listings}

\begin{document}

\title{CS 260}
\author{Daniel Lopez}
\maketitle

\date{29 June 2017}

\section{This is a section header.}

\subsection{This is a subsection header.}

\begin{itemize}
	\item First item.
	\item Second item.
	\item Third item.
	\item Fifth item.
	\item Fourth item.
\end{itemize}

\begin{enumerate}
	\item First item.
	\item Second item.
	\item Third item.
	\item Fourth item.
	\item Fifth item.
\end{enumerate}

This is a paragraph of text. There isn't much \textit{useful} information here, just some italics. But, it is a paragraph nonetheless.\par
Here is another paragraph. This one actually has some useful information. Or...\textbf{does} it?\par

At long last, here is a sentence with something useful, as shown by the equation $a^2 = b^2 + c^2$, which is the Pythagorean Theorem.\par

\begin{equation}
e^{\pi i} + 1 = 0
\end{equation}

\begin{align}
	2+2=&4
	||
	5x+3=&2y
	||
	x+y+z=325
\end{align}

\begin{lstlisting}
x = 0
arr = [1,2,3,4,5]
for x in len(arr):
	x = x + 1
	arr[x] = 0
print("array values set to 0")
\end{lstlisting}

\begin{algorithm}[H]
	\KwData{One Integer as input}
	A=1,2,3,4,5\;
	\For{i=2,i $<$ 10; i++}
	{
		A[i]=i*2
	}
\end{algorithm}

\end{document}
